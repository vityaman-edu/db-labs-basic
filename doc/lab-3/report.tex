\documentclass{article}

\usepackage[utf8]{inputenc}
\usepackage[russian]{babel}
\usepackage[a4paper, margin=1in]{geometry}
\usepackage{graphicx}
\usepackage{amsmath}
\usepackage{wrapfig}
\usepackage{multirow}
\usepackage{mathtools}
\usepackage{pgfplots}
\usepackage{pgfplotstable}
\usepackage{setspace}
\usepackage{changepage}
\usepackage{caption}
\usepackage{csquotes}
\usepackage{hyperref}
\usepackage{listings}

\pgfplotsset{compat=1.18}
\hypersetup{
  colorlinks = true,
  linkcolor  = blue,
  filecolor  = magenta,      
  urlcolor   = darkgray,
  pdftitle   = {
    database-report-3-smirnov-victor-p33131
  },
}

\definecolor{codegreen}{rgb}{0,0.6,0}
\definecolor{codegray}{rgb}{0.5,0.5,0.5}
\definecolor{codepurple}{rgb}{0.58,0,0.82}
\definecolor{backcolour}{rgb}{0.99,0.99,0.99}

\lstdefinestyle{codestyle}{
  backgroundcolor=\color{backcolour},   
  commentstyle=\color{codegreen},
  keywordstyle=\color{magenta},
  numberstyle=\tiny\color{codegray},
  stringstyle=\color{codepurple},
  basicstyle=\ttfamily\footnotesize,
  breakatwhitespace=false,         
  breaklines=true,                 
  captionpos=b,                    
  keepspaces=true,                 
  numbers=left,                    
  numbersep=5pt,                  
  showspaces=false,                
  showstringspaces=false,
  showtabs=false,                  
  tabsize=2
}

\graphicspath{ {./img/} }

\lstset{style=codestyle}

\begin{document}

\begin{titlepage}
    \begin{center}
        \begin{spacing}{1.4}
            \large{Университет ИТМО} \\
            \large{Факультет программной инженерии и компьютерной техники} \
        \end{spacing}
        \vfill
        \textbf{
            \huge{Базы данных.} \\
            \huge{Лабораторная работа №3.} \\
        }
    \end{center}
    \vfill
    \begin{center}
        \begin{tabular}{r l}
            Группа:  & P33131                  \\
            Студент: & Смирнов Виктор Игоревич \\
            Вариант: & 3313103
        \end{tabular}
    \end{center}
    \vfill
    \begin{center}
        \begin{large}
            2023
        \end{large}
    \end{center}
\end{titlepage}

\section*{Ключевые слова}

База данных

\tableofcontents

\section{Цель работы}

Научиться писать запросы к БД.

\section{Задача}

Составить запросы на языке SQL (пункты 1-7).

Сделать запрос для получения атрибутов из указанных таблиц, 
применив фильтры по указанным условиям:
Н\_ТИПЫ\_ВЕДОМОСТЕЙ, Н\_ВЕДОМОСТИ.
Вывести атрибуты: Н\_ТИПЫ\_ВЕДОМОСТЕЙ.ИД, Н\_ВЕДОМОСТИ.ЧЛВК\_ИД.
Фильтры (AND):
a) Н\_ТИПЫ\_ВЕДОМОСТЕЙ.НАИМЕНОВАНИЕ = Перезачет.
b) Н\_ВЕДОМОСТИ.ДАТА > 2022-06-08.
c) Н\_ВЕДОМОСТИ.ДАТА = 2010-06-18.
Вид соединения: RIGHT JOIN.
Сделать запрос для получения атрибутов из указанных таблиц, применив фильтры по указанным условиям:
Таблицы: Н\_ЛЮДИ, Н\_ВЕДОМОСТИ, Н\_СЕССИЯ.
Вывести атрибуты: Н\_ЛЮДИ.ИД, Н\_ВЕДОМОСТИ.ИД, Н\_СЕССИЯ.ИД.
Фильтры (AND):
a) Н\_ЛЮДИ.ИМЯ > Николай.
b) Н\_ВЕДОМОСТИ.ДАТА < 2010-06-18.
Вид соединения: RIGHT JOIN.
Вывести число студентов вечерней формы обучения, которые не имеет отчества.
Ответ должен содержать только одно число.
Найти группы, в которых в 2011 году было ровно 5 обучающихся студентов на ФКТИУ.
Для реализации использовать подзапрос.
Выведите таблицу со средними оценками студентов группы 4100 (Номер, ФИО, Ср\_оценка), у которых средняя оценка меньше средней оценк(е|и) в группе 1101.
Получить список студентов, зачисленных до первого сентября 2012 года на первый курс очной формы обучения. В результат включить:
номер группы;
номер, фамилию, имя и отчество студента;
номер и состояние пункта приказа;
Для реализации использовать соединение таблиц.
Вывести список студентов, имеющих одинаковые фамилии, но не совпадающие даты рождения.

\section{Вывод}


% \begin{thebibliography}{9}
    
% \end{thebibliography}

\end{document}